\documentclass[a4paper,11pt,oneside,pdftex]{scrartcl}

\usepackage{ucs}
\usepackage[utf8x]{inputenc}
\usepackage[english,german]{babel}
\usepackage{fontenc}
\usepackage{graphicx}

%\usepackage[pdftex]{hyperref}
\usepackage{listings}
\usepackage{color}
 
\definecolor{dkgreen}{rgb}{0,0.6,0}
\definecolor{gray}{rgb}{0.5,0.5,0.5}
\definecolor{mauve}{rgb}{0.58,0,0.82}
 
\newcounter{enumi_saved}
\newcommand{\breaktable}{%
 \setcounter{enumi_saved}{\value{enumi}}
 \end{enumerate}
 \begin{enumerate}
 \setcounter{enumi}{\value{enumi_saved}}
}

\lstset{ %
  basicstyle=\footnotesize,           % the size of the fonts that are used for the code
  numbers=left,                   % where to put the line-numbers
  numberstyle=\tiny\color{gray},  % the style that is used for the line-numbers
  stepnumber=2,                   % the step between two line-numbers. If it's 1, each line 
                                  % will be numbered
  numbersep=5pt,                  % how far the line-numbers are from the code
  backgroundcolor=\color{white},      % choose the background color. You must add \usepackage{color}
  showspaces=false,               % show spaces adding particular underscores
  showstringspaces=false,         % underline spaces within strings
  showtabs=false,                 % show tabs within strings adding particular underscores
  frame=single,                   % adds a frame around the code
  rulecolor=\color{black},        % if not set, the frame-color may be changed on line-breaks within not-black text (e.g. commens (green here))
  tabsize=2,                      % sets default tabsize to 2 spaces
  captionpos=b,                   % sets the caption-position to bottom
  breaklines=true,                % sets automatic line breaking
  breakatwhitespace=false,        % sets if automatic breaks should only happen at whitespace
   keywordstyle=\color{blue},          % keyword style
  commentstyle=\color{dkgreen},       % comment style
  stringstyle=\color{mauve},         % string literal style
  escapeinside={\%*}{*)},            % if you want to add LaTeX within your code
  morekeywords={*,...}               % if you want to add more keywords to the set
}

\begin{document}
 \title{Tutorial: Adding new messages in Protobuf}
 \date{}
\author{Michael Schreier}
\maketitle
\begin{abstract}
 This document intends to give an advice for adding new messages in Protobuf (with C++ in ANNarchy) and Protobuf-Net (with C\# in SimpleNetwork).
 I will go through the process by example by adding the new message \emph{MsgAgentGrapPos}, but neither explain the message syntax nor how to use it. 
\end{abstract}


\section{Adding the message in ANNarchy}

\begin{enumerate}
 \item Open \emph{ANNarchy-2.2/protobuf/MsgObject.proto}.
 \item Add the listing
\begin{lstlisting}
message MsgAgentGraspPos {
  required int32 actionID = 1;
  required float targetX = 2;
  required float targetY = 3;
}
\end{lstlisting}
 \item Add a new line in the existing \emph{MsgObject}.
\begin{lstlisting}
optional MsgAgentGraspPos msgAgentGraspPos = 13;
\end{lstlisting}
Here I added it as the 13th entity, so note that each new message must have a new ID.
\item Run \begin{lstlisting}
protoc MsgObject.proto --cpp_out=.
\end{lstlisting}
 \item Copy the resulting \emph{MsgObject.pb.cc} and \emph{MsgObject.pb.h} in ANNarchys \emph{include}
and \emph{src} directory.
\item Optionally: Open \emph{MsgObject.pb.cc} and customize the line 
\begin{lstlisting}
#include "MsgObject.pb.h"
\end{lstlisting}
for your purpose or change the include directories in Eclipse.
\item Now the new message is ready to be used.


\item The last step is adding a new public member function to the \emph{AnnarProtoSend} class. This function
will encapsulate the construction and sending of the \emph{MsgObject}.
Don't forget to add the function in both, \emph{AnnarProtoSend.h} and \emph{AnnarProtoSend.cpp}.
\begin{lstlisting}
int AnnarProtoSend::sendGraspPos(float targetX, float targetY)
{
	MsgObject * unit = new MsgObject();
	int id = rand();
	unit->mutable_msgagentgrasppos()->set_actionid(id);
	unit->mutable_msgagentgrasppos()->set_targetx(targetX);
	unit->mutable_msgagentgrasppos()->set_targety(targetY);

	mutex = true;
	sentQueue.push(unit);
	mutex = false;
	return id;
}
\end{lstlisting}
\end{enumerate}
\section{Adding the message in SimpleNetwork (with Protobuf-Net)}

\begin{enumerate}
\item Open the \emph{SimpleNetwork}-Project.
\item Add a new class \emph{MsgAgentGraspPos} in the \emph{Msg} folder. This will create the new file \emph{MsgAgentGraspPos.cs}.
\item Edit \emph{MsgAgentGraspPos.cs}.
\begin{lstlisting}
using System;
using System.Collections.Generic;
using System.Linq;
using System.Text;
using ProtoBuf;

namespace SimpleNetwork
{
    [Serializable, ProtoContract]
    public class MsgAgentGraspPos
    {
        [ProtoMember(1, IsRequired = true)]
        public Int32 actionID;       
        [ProtoMember(2, IsRequired = true)]
        public float targetX;    
        [ProtoMember(3, IsRequired = true)]
        public float targetY;
    }
} 
\end{lstlisting}
\item Edit \emph{MsgObject.cs} and add
\begin{lstlisting}
        [ProtoMember(13)]
        public MsgAgentGraspPos msgAgentGraspPos;   
\end{lstlisting}
Here, same as above. Each new entity must have a new ID.
\item Build \emph{SimpleNetwork}-Project.
\item Now the assembly \emph{SimpleNetwork.dll} can be used with the new message\\\emph{MsgAgentGraspPos}.

\item In our Project "Kinderzimmer" (Children's room), the \emph{update} functions of \emph{BehaviourScript} and \emph{AgentScript} are responsible for the message handling. For each protobuf message, there also exists a protected virtual function, which will be customized in derived classes of \emph{AgentScript}.

Here is the  \emph{update} of \emph{AgentScript}
\begin{lstlisting}
void Update()
{
	#region handle incoming msg
	if(MySimpleNet != null){
	
	if (MySimpleNet.MsgAvailable()) {

		MsgObject NextMsg = MySimpleNet.Receive();

		[...]
           	
		if(NextMsg.msgAgentGraspPos != null) {
                	processMsgAgentGrapPos( NextMsg.msgAgentGraspPos );
          	}

		[...]

	}
}\end{lstlisting}
and the message handling function for \emph{MsgAgentGraspPos}
\begin{lstlisting}
protected virtual void processMsgAgentGrapPos( MsgAgentGraspPos msg )
    {
        Debug.Log(String.Format("AgentGraspPos recived ({0:f},{1:f})",
                        msg.targetX,
                        msg.targetY));
    }
\end{lstlisting}





For a new protobuf message, the if-switch in the \emph{update} function must be extented and a new protected virtual function added.



\end{enumerate}

\newpage
\section{Appendix: Sending in C\#/Receiving in C++}

I explained in the first two sections the work process for adding the \emph{MsgAgentGraspPos} message. This section explains, what to do, if you want to receive data in C++ and send data from C\# with a message (e.g. the \emph{MsgGridPosition} message).

\subsection*{Sending data from C\#}

Sending protobuf data is simple. For example \emph{MsgGridPosition} is sent in the \emph{AgentScript} \emph{Update()} with:
\begin{lstlisting}
void Update()
{
	[...]
	if (SendGridPosition) {
		MySimpleNet.Send(new MsgGridPosition(){
			targetX = gameObject.transform.position.x,
			targetY = gameObject.transform.position.y,
			targetZ = gameObject.transform.position.z
		});
	}
	[...]
}
\end{lstlisting}

\subsection*{Receiving data in C++}
\begin{enumerate}
\item Add and initialize member variables for the \emph{MsgGridPosition} message in\\ \mbox{AnnarProtoReceive}.

AnnarProtoReceive.h
\begin{lstlisting}
class AnnarProtoReceive{
	[...]
	bool validGridsensor_;
	float targetX_;          
	float targetY_;        
	float targetZ_;
	[...]
}
\end{lstlisting}
AnnarProtoReceive.cpp
\begin{lstlisting}
AnnarProtoReceive::AnnarProtoReceive(int socketVR, int socketAgent)
{
	[...]

	validGridsensor_ = false;
	targetX_ = 0;
	targetY_ = 0;
	targetZ_ = 0;

	[...]
}
\end{lstlisting}

\item Add a new getter function for this message.

AnnarProtoRecive.cpp
\begin{lstlisting}
bool AnnarProtoReceive::getGridsensorData(float& targetX, float& targetY, float& targetZ)
{
	if(!validGridsensor_)
		return false;
	waitForMutexUnlock();

	targetX = targetX_;
	targetY = targetY_;
	targetZ = targetZ_;
	return true;
}

\end{lstlisting}

\item Add a new if-switch to the \emph{AnnarProtoReceive::storeData} function.

\begin{lstlisting}
void AnnarProtoReceive::storeData(long int dataLength)
{
[...]
	if(tmp->has_msggridposition())
	{
		assert(tmp->msggridposition().has_targetx());
		mutex = true;
		targetX_ = tmp->msggridposition().targetx();
		targetY_ = tmp->msggridposition().targety();
		targetZ_ = tmp->msggridposition().targetz();
		validGridsensor_ = true;
		mutex = false;
	}
[...]
}
\end{lstlisting}
\item Now the data of a \emph{MsgGridPosition} message can be received by
\begin{lstlisting}
receiver->getGridsensorData(agentX, agentY, agentZ)
\end{lstlisting}
\end{enumerate}



\end{document}
